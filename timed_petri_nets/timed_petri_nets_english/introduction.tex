\section{Introduction}
	% no \IEEEPARstart
	
	The evolution of the processors is consequence of the greater integration and composition of different 
	types of functionalities integrated into a single processor. Today, the availability of transistors 
	has made possible to integrate several processor cores on a single chip, which has resulted in 
	the development of Multi-Core technology \cite{hennessypatterson}.
	\\
	
	Diminishing returns of Instruction Level Parallelism (ILP) and the cost of the increase of frequency, 
	mainly due to power limitations (suggests that a 1\% increase in clock speed results in a power 
	increase of 3\%) \cite{gspn}, leads to the use of multicore processors to improve performance. 
	This increase deficiency results in lower run times, lower consumption, lower energy density, 
	lower latency and higher bandwidth inter-core communications.
	\\
	
	Furthermore, the heterogeneous multi-core systems have the advantage of employing specialized cores, 
	each of them designed for specific tasks. That is, optimized for a particular need. These processors 
	have the ability to use the available hardware resources when they are specifically required by 
	the software  \cite{SriramBhattacharyya}.
	\\
	
	In order to increase performance, these systems make use of multi-threading and/or multi-tasking 
	allowing take advantage of the multi-cores. However, it takes more effort to design applications 
	because they must provide solution to the problems of concurrent systems. Thus, the parallel 
	programming is essential to improve the performance of the software in these systems.
    \\
	
	At the Computer Architectures Laboratory of the FCEFyN-UNC a Petri processor has been developed 
	to directly execute ordinary Petri Nets. In this article, we present a new Petri Nets Processor 
	capable to execute Timed Petri Nets and to be programmed directly with the vectors and matrixes 
	that define the system and its state.
	