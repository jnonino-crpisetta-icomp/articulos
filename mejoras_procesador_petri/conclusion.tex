%%%%%%%%%%%%%%%%%%%%%%%%%%%%%%%%%%%%%%%%%%%%%%%%%%%%%%%%%%%%%%%%%%%%%%%%%%%%%%%%%%%%%
%																					%
%	TRABAJO: Paper Mejoras en el procesador de Redes de Petri						%
%																					%
%		Titulo: 	Soft Core parametrizable con procesamiento de Redes de Petri	%
%																					%
%		Autores:	Julian Nonino													%
%					Carlos Renzo Pisetta											%
%					Orlando Micolini												%
%																					%
%	Seccion: Conclusion																%	
%	Archivo: conclusion.tex															%
%																					%
%%%%%%%%%%%%%%%%%%%%%%%%%%%%%%%%%%%%%%%%%%%%%%%%%%%%%%%%%%%%%%%%%%%%%%%%%%%%%%%%%%%%%

\section{Conclusiones}

	En �ste trabajo se present� un nuevo procesador de Petri y se realiz� un IP Core parametrizable 
	que verifica y resuelve una ejecuci�n en dos ciclos desde que entra en la cola de entrada, hasta 
	que es depositado en la cola de salida. 

	La nueva capacidad de interrumpir permite un nuevo m�todo de espera no activo, sin eliminar el 
	m�todo por consulta que utilizaba el dise�o anterior. 
	Las transiciones automaticas permite, la evoluci�n del sistema sin la necesidad de pedir un disparo 
	desde el exterior, y las cotas permite acotar cada plaza a un n�mero determinado de tokens simplemente 
	cargando los datos desde el software, sin la necesidad de modificar el HDL.