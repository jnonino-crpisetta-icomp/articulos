%%%%%%%%%%%%%%%%%%%%%%%%%%%%%%%%%%%%%%%%%%%%%%%%%%%%%%%%%%%%%%%%%%%%%%%%%%%%%%%%%%%%%
%																					%
%	TRABAJO: Paper Mejoras en el procesador de Redes de Petri						%
%																					%
%		Titulo: 	Soft Core parametrizable con procesamiento de Redes de Petri	%
%																					%
%		Autores:	Julian Nonino													%
%					Carlos Renzo Pisetta											%
%					Orlando Micolini												%
%																					%
%	Seccion: Resumen/Abstract														%	
%	Archivo: resumen.tex															%
%																					%
%%%%%%%%%%%%%%%%%%%%%%%%%%%%%%%%%%%%%%%%%%%%%%%%%%%%%%%%%%%%%%%%%%%%%%%%%%%%%%%%%%%%%

\begin{abstract}
%\boldmath
	Este desarrollo, mediante el empleo de Redes de Petri, plante� una soluci�n a problemas 
    de sincronizaci�n en arquitecturas multicore. 
	El implemento de un procesador re-programable en tiempo de ejecuci�n que utilice estos 
	formalismos en hardware con uso de FPGA, permiti� dise�ar modificaciones en la arquitectura 
	y expandir funcionalidades del IP Core, las cuales han sido descriptas en este trabajo. 
\end{abstract}
% IEEEtran.cls defaults to using nonbold math in the Abstract.
% This preserves the distinction between vectors and scalars. However,
% if the journal you are submitting to favors bold math in the abstract,
% then you can use LaTeX's standard command \boldmath at the very start
% of the abstract to achieve this. Many IEEE journals frown on math
% in the abstract anyway.

% Note that keywords are not normally used for peerreview papers.
\begin{IEEEkeywords}
Petri Nets, MultiCore, FPGA, IP Core.
\end{IEEEkeywords}