\section{Conclusion and Contributions}

	In this paper, a Time Petri Processor is developed, which decouples the concurrency from sequential 
	processing, it has the following particularities:
	\begin{itemize}
  		\item On tasks, where measurements have been done, the processor allows synchronization of 
  				threads, with improvements up to 70\%.
		\item There is a direct relationship between the graph and the processor program, since this 
				is programmed with the state equation's matrixes and vectors.
		\item Multiple are admitted shots simultaneously in the same transition.
		\item Allows programming priorities since the shots are solved in parallel and are selected 
				according to priorities module.
		\item Decides if the shot can be executed or not in 2 clock cycles.
		\item The system programming is easier to do, since the processes are decoupled from the concurrency.
		\item This processor can be programmed at run time, thus it is possible to decrease the size 
				of the matrix in hardware by using spatial and temporal locality.
	\end{itemize}
	
	The difficulty of this implementation is because of the growth of the resources needed by the 
	increase of places and transitions. This implies that is difficult to implement a system for 
	dimensions greater than 32x32 in ZedBoard, to mitigate this difficulty new designs have been 
	proposed and are being worked on, these are: Petri Net Processor with pipeline architecture and 
	support for Hierarchical Petri Nets.