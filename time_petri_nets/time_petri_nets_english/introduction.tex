\section{Introduction}
	\IEEEPARstart{T}{he} computer systems are complex as much its structure as its behavior, even 
	more when they have a great number of states and many combinations of data and input events.

Develop solutions of complex and critical systems for give a solution a real-time systems have 
 problems such as: the inherent complexity of the specification, the coordination of concurrent 
 tasks, the lack of portable algorithms, standardized environments, software and development tools.

And taking into account unambiguous trends in the hardware design, which indicate that one processor
 may not be able to keep pace with increase performance. Therefore the evolution of the processors 
 is consequence of the greater integration and composition of different types of functionalities 
 integrated into a single processor. Even more today, the availability of transistors has made 
 possible to integrate several processor cores on a single chip, which has resulted in the 
 development of Multi-Core technology \cite{hennessypatterson}.
 
Diminishing returns of Instruction Level Parallelism (ILP) and the cost of the increase of frequency,
 mainly due to power limitations (suggests that a 1\% increase in clock speed results in a power 
 increase of 3\%) \cite{domeika}, leads to the use of multicore processors to improve performance. 
 This increase deficiency results in lower run times, lower consumption, lower energy density, lower
 latency and higher bandwidth inter-core communications.
    
Therefore multi-core processors are a proposal to obtain higher performance. This mainly involves 
 lower execution time, energy consumption, energy density, latency and more bandwidth inter-core 
 communications. Furthermore, the heterogeneous multi-core systems have the advantage of employing 
 specialized cores, each of them designed for specific tasks. That is, optimized for a particular 
 need. These processors have the ability to use the available hardware resources when they are 
 specifically required by the software \cite{sriram}.

In order to increase performance, these systems make use of multi-threading and/or multi-tasking 
 allowing take advantage of the multi-cores. However, it takes more effort to design applications 
 because they must provide solution to the problems of concurrent systems.

That is the reason why with these processors, the parallel programming is essential for improving 
 the performance in all segments of software development and even more so in the segment of real-time
 systems.
 
At the Computer Architectures Laboratory of the FCEFyN-UNC a Petri processor has been developed to 
 directly execute ordinary Petri Nets. In this article, we present a new Petri Nets Processor capable
 to execute Time Petri Nets and to be programmed directly with the vectors and matrixes that define 
 the system and its state.
	

