%	TITLE
	% paper title
	% can use linebreaks \\ within to get better formatting as desired
		\title{Execution of Timed Petri Nets with IP cores}

%	AUTHORS
	% author names and affiliations
	% use a multiple column layout for up to three different affiliations

	% conference papers do not typically use \thanks and this command
	% is locked out in conference mode. If really needed, such as for
	% the acknowledgment of grants, issue a \IEEEoverridecommandlockouts
	% after \documentclass

	% for over three affiliations, or if they all won't fit within the width
	% of the page, use this alternative format:
	% 
	\author{
		\IEEEauthorblockN{Orlando Micolini}
		\IEEEauthorblockA{
			Laboratorio de Arquitectura \\ de Computadoras\\
			FCEFyN-UNC\\
			C�rdoba, Argentina\\
			Email: omicolini@compuar.com
		}	
	\and
		\IEEEauthorblockN{Julian Nonino}
		\IEEEauthorblockA{
			Laboratorio de Arquitectura \\ de Computadoras\\
			FCEFyN-UNC\\
			C�rdoba, Argentina\\
			Email: noninojulian@gmail.com
		}	
		\and
		\IEEEauthorblockN{Carlos Renzo Pisetta}
		\IEEEauthorblockA{
			Laboratorio de Arquitectura \\ de Computadoras\\
			FCEFyN-UNC\\
			C�rdoba, Argentina\\
			Email: renzopisetta@gmail.com
		}	
	}

	% use for special paper notices
	%\IEEEspecialpapernotice{(Invited Paper)}

	% make the title area
		\maketitle
		
	% For peer review papers, you can put extra information on the cover
	% page as needed:
	% \ifCLASSOPTIONpeerreview
	% \begin{center} \bfseries EDICS Category: 3-BBND \end{center}
	% \fi
	%
	% For peerreview papers, this IEEEtran command inserts a page break and
	% creates the second title. It will be ignored for other modes.
		\IEEEpeerreviewmaketitle