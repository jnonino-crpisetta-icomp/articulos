%%%%%%%%%%%%%%%%%%%%%%%%%%%%%%%%%%%%%%%%%%%%%%%%%%%%%%%%%%%%%%%%%%%%%%%%%%%%%%%%%%%%%
%																					%
%	TRABAJO: Paper Redes de Petri con Tiempo										%
%																					%
%		Titulo: 	Ejecucion de Redes de Petri con Tiempo							%
%																					%
%		Autores:	Julian Nonino													%
%					Carlos Renzo Pisetta											%
%					Orlando Micolini												%
%																					%
%	Seccion: Introduccion															%	
%	Archivo: introduccion.tex														%
%																					%
%%%%%%%%%%%%%%%%%%%%%%%%%%%%%%%%%%%%%%%%%%%%%%%%%%%%%%%%%%%%%%%%%%%%%%%%%%%%%%%%%%%%%
%	REVISIONES																		%
%																					%
%		18/10/2012																	%
%			Julian Nonino															%
%				Creacion de este archivo											%
%																					%
%%%%%%%%%%%%%%%%%%%%%%%%%%%%%%%%%%%%%%%%%%%%%%%%%%%%%%%%%%%%%%%%%%%%%%%%%%%%%%%%%%%%%

\section{Introducci�n}
	% The very first letter is a 2 line initial drop letter followed
	% by the rest of the first word in caps.
	% 
	% form to use if the first word consists of a single letter:
	% \IEEEPARstart{A}{demo} file is ....
	% 
	% form to use if you need the single drop letter followed by
	% normal text (unknown if ever used by IEEE):
	% \IEEEPARstart{A}{}demo file is ....
	% 
	% Some journals put the first two words in caps:
	% \IEEEPARstart{T}{his demo} file is ....
	% 
	% Here we have the typical use of a "T" for an initial drop letter
	% and "HIS" in caps to complete the first word.
		%\IEEEPARstart{T}{his} demo file is intended to serve as a ``starter file''
		%for IEEE journal papers produced under \LaTeX\ using
		%IEEEtran.cls version 1.7 and later.
		% You must have at least 2 lines in the paragraph with the drop letter
		% (should never be an issue)
		%I wish you the best of success.
		
	\IEEEPARstart{D}{esde} hace tiempo, los sistemas de computaci�n son multiprogramados, esto significa que en 
	un momento dado existen m�ltiples procesos cooperando por un fin com�n y/o compitiendo por recursos limitados.
	En un sistema monoprocesador, se produce una intercalaci�n de instrucciones de diversos procesos concurrentes 
	aparentando una ejecuci�n paralela.
	Esto, tiene el problema de la sincronizaci�n para el uso de recursos, en la mayor�a de los casos un proceso no 
	podr� compartir un recurso mientras lo usa, por lo tanto, los dem�s deben bloquearse y esperar hasta que el 
	recurso sea liberado. Adem�s, si los procesos comparten datos, se debe asegurar que solo uno de ellos lo 
	modifique en un mismo instante. Por ello, para que el sistema funcione correctamente y se conserve la 
	integridad de los datos se deben utilizar diversos mecanismos de sincronizaci�n y de exclusi�n mutua.
	Sumado a esto, en la actualidad, la mayor�a de los sistemas incluyen m�ltiples procesadores, por ende, 
	los problemas anteriormente mencionados se multiplican ya que la ejecuci�n paralela es real e intercalada.

	Los problemas generados por la ejecuci�n concurrente son: la necesidad de exclusi�n mutua, condiciones de 
	sincronizaci�n, interbloqueos e inanici�n. Para detectar estos problemas se debe modelar el sistema. 
	Una de las herramientas para modelar procesos concurrentes son las m�quinas de estado. El problema de esta 
	herramienta, es la distancia que existe entre el modelo y la implementaci�n, lo que en consecuencia, nos 
	dificulta asegurar que la implementaci�n cumpla con las propiedades validadas y verificadas en el modelo. 

	En el Laboratorio de Arquitecturas de Computadoras desde hace tiempo se trabaja con otra herramienta para 
	modelar sistemas concurrentes. �sta, es una herramienta gr�fica con una base matem�tica formal y se conoce 
	como Redes de Petri.
	Las Redes de Petri, logran modelar los diferentes estados de los sistemas reactivos y las posibles 
	transiciones entre ellos. Este modelo gr�fico puede ser traducido a una ecuaci�n de estado que definir� el 
	estado siguiente del sistema seg�n el estado actual y de las transiciones que desean dispararse en forma 
	algebraica. Pero el hecho m�s importante, es que estas redes no solo sirven para la simulaci�n sino, que 
	adem�s, pueden ser ejecutadas, con lo cual, la distancia entre el modelo y la implementaci�n no existe. 
	Modelando con Redes de Petri, el funcionamiento y las propiedades del sistema pueden ser asegurados a�n 
	antes de la implementaci�n.
	Hay que destacar la importancia del estudio de las Redes de Petri, ya que en los �ltimos diez a�os se 
	han realizado 10294 publicaciones con referato.
		
%\hfill mds
%\hfill January 11, 2007

	\subsection{Objetivos}
		\subsubsection{Objetivo principal}
			El objetivo principal de este trabajo es dise�ar e implementar un procesador de Redes de Petri
			que sea capaz de procesar Redes de Petri Temporales para la  sem�ntica \textbf{Redes de Petri con Tiempo}. 
			Manteniendo la condici�n de programaci�n directa entre el modelo y la implementaci�n.

% needed in second column of first page if using \IEEEpubid
%\IEEEpubidadjcol

		\subsubsection{Objetivos secundarios}
			Los objetivos secundarios de este trabajo son:
			\begin{itemize}
   				\item Analizar las Redes de Petri Temporales con el fin de evaluar su implementaci�n 
   						por hardware.
   				\item Redise�ar el procesador de Redes de Petri con el objetivo de posibilitar la 
   						inserci�n de par�metros temporales.
   				\item Implementar el procesador de Redes de Petri como un IP core.
   			\end{itemize}	


% An example of a floating figure using the graphicx package.
% Note that \label must occur AFTER (or within) \caption.
% For figures, \caption should occur after the \includegraphics.
% Note that IEEEtran v1.7 and later has special internal code that
% is designed to preserve the operation of \label within \caption
% even when the captionsoff option is in effect. However, because
% of issues like this, it may be the safest practice to put all your
% \label just after \caption rather than within \caption{}.
%
% Reminder: the "draftcls" or "draftclsnofoot", not "draft", class
% option should be used if it is desired that the figures are to be
% displayed while in draft mode.
%
%\begin{figure}[!t]
%\centering
%\includegraphics[width=2.5in]{myfigure}
% where an .eps filename suffix will be assumed under latex, 
% and a .pdf suffix will be assumed for pdflatex; or what has been declared
% via \DeclareGraphicsExtensions.
%\caption{Simulation Results}
%\label{fig_sim}
%\end{figure}

% Note that IEEE typically puts floats only at the top, even when this
% results in a large percentage of a column being occupied by floats.


% An example of a double column floating figure using two subfigures.
% (The subfig.sty package must be loaded for this to work.)
% The subfigure \label commands are set within each subfloat command, the
% \label for the overall figure must come after \caption.
% \hfil must be used as a separator to get equal spacing.
% The subfigure.sty package works much the same way, except \subfigure is
% used instead of \subfloat.
%
%\begin{figure*}[!t]
%\centerline{\subfloat[Case I]\includegraphics[width=2.5in]{subfigcase1}%
%\label{fig_first_case}}
%\hfil
%\subfloat[Case II]{\includegraphics[width=2.5in]{subfigcase2}%
%\label{fig_second_case}}}
%\caption{Simulation results}
%\label{fig_sim}
%\end{figure*}
%
% Note that often IEEE papers with subfigures do not employ subfigure
% captions (using the optional argument to \subfloat), but instead will
% reference/describe all of them (a), (b), etc., within the main caption.


% An example of a floating table. Note that, for IEEE style tables, the 
% \caption command should come BEFORE the table. Table text will default to
% \footnotesize as IEEE normally uses this smaller font for tables.
% The \label must come after \caption as always.
%
%\begin{table}[!t]
%% increase table row spacing, adjust to taste
%\renewcommand{\arraystretch}{1.3}
% if using array.sty, it might be a good idea to tweak the value of
% \extrarowheight as needed to properly center the text within the cells
%\caption{An Example of a Table}
%\label{table_example}
%\centering
%% Some packages, such as MDW tools, offer better commands for making tables
%% than the plain LaTeX2e tabular which is used here.
%\begin{tabular}{|c||c|}
%\hline
%One & Two\\
%\hline
%Three & Four\\
%\hline
%\end{tabular}
%\end{table}


% Note that IEEE does not put floats in the very first column - or typically
% anywhere on the first page for that matter. Also, in-text middle ("here")
% positioning is not used. Most IEEE journals use top floats exclusively.
% Note that, LaTeX2e, unlike IEEE journals, places footnotes above bottom
% floats. This can be corrected via the \fnbelowfloat command of the
% stfloats package.