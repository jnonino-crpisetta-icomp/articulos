%%%%%%%%%%%%%%%%%%%%%%%%%%%%%%%%%%%%%%%%%%%%%%%%%%%%%%%%%%%%%%%%%%%%%%%%%%%%%%%%%%%%%
%																					%
%	TRABAJO: Paper Redes de Petri con Tiempo										%
%																					%
%		Titulo: 	Ejecucion de Redes de Petri con Tiempo							%
%																					%
%		Autores:	Juli�n Nonino													%
%					Carlos Renzo Pisetta											%
%					Orlando Micolini												%
%																					%
%	Seccion: Resumen/Abstract														%	
%	Archivo: resumen.tex															%
%																					%
%%%%%%%%%%%%%%%%%%%%%%%%%%%%%%%%%%%%%%%%%%%%%%%%%%%%%%%%%%%%%%%%%%%%%%%%%%%%%%%%%%%%%
%	REVISIONES																		%
%																					%
%		18/10/2012																	%
%			Julian Nonino															%
%				Creacion de este archivo											%
%		21/10/2012																	%
%			Julian Nonino															%
%				Escritura del abstract y de las palabras clave						%
%																					%
%%%%%%%%%%%%%%%%%%%%%%%%%%%%%%%%%%%%%%%%%%%%%%%%%%%%%%%%%%%%%%%%%%%%%%%%%%%%%%%%%%%%%

\begin{abstract}
%\boldmath
En este trabajo, se presenta un an�lisis de las Redes de Petri con Tiempo. 
De esta manera, se puede aprovechar el poder de las Redes de Petri para modelar sistemas de tiempo real, 
verificando formalmente todas sus propiedades.

Posteriormente, se presenta el desarrollo de un IP cores, capaz de ejecutar Redes de Petri con
Tiempo. De esta manera, es posible realizar la implementaci�n del sistema utilizando este IP core 
asegurando que todas las propiedades del modelo se verifican en el sistema real.
\end{abstract}
% IEEEtran.cls defaults to using nonbold math in the Abstract.
% This preserves the distinction between vectors and scalars. However,
% if the journal you are submitting to favors bold math in the abstract,
% then you can use LaTeX's standard command \boldmath at the very start
% of the abstract to achieve this. Many IEEE journals frown on math
% in the abstract anyway.

% Note that keywords are not normally used for peerreview papers.
\begin{IEEEkeywords}
Redes de Petri Temporales, IP core, FPGA.
\end{IEEEkeywords}