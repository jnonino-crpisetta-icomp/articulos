%%%%%%%%%%%%%%%%%%%%%%%%%%%%%%%%%%%%%%%%%%%%%%%%%%%%%%%%%%%%%%%%%%%%%%%%%%%%%%%%%%%%%
%																					%
%	TRABAJO: Paper Redes de Petri con Tiempo										%
%																					%
%		Titulo: 	Ejecucion de Redes de Petri con Tiempo							%
%																					%
%		Autores:	Julian Nonino													%
%					Carlos Renzo Pisetta											%
%					Orlando Micolini												%
%																					%
%	Seccion: Definicion de titulos y autores										%	
%	Archivo: titulo_autores.tex														%
%																					%
%%%%%%%%%%%%%%%%%%%%%%%%%%%%%%%%%%%%%%%%%%%%%%%%%%%%%%%%%%%%%%%%%%%%%%%%%%%%%%%%%%%%%
%	REVISIONES																		%
%																					%
%		18/10/2012																	%
%			Julian Nonino															%
%				Creacion de este archivo y escritura del titulo y autores			%
%																					%
%%%%%%%%%%%%%%%%%%%%%%%%%%%%%%%%%%%%%%%%%%%%%%%%%%%%%%%%%%%%%%%%%%%%%%%%%%%%%%%%%%%%%

	%	TITULO
		% paper title
		% can use linebreaks \\ within to get better formatting as desired
			\title{IP cores para la Ejecuci�n de \\Redes de Petri Temporales en FPGA}

	%	AUTORES
		% author names and IEEE memberships
		% note positions of commas and nonbreaking spaces ( ~ ) LaTeX will not break
		% a structure at a ~ so this keeps an author's name from being broken across
		% two lines.
		% use \thanks{} to gain access to the first footnote area
		% a separate \thanks must be used for each paragraph as LaTeX2e's \thanks
		% was not built to handle multiple paragraphs
		%
			\author{Juli�n~Nonino,~\IEEEmembership{Member,~IEEE,}
		        Carlos~Renzo~Pisetta,~\IEEEmembership{Member,~IEEE,}
		        and~Orlando~Micolini,~\IEEEmembership{Member,~IEEE}% <-this % stops a space
			}
		%
		% note the % following the last \IEEEmembership and also \thanks - 
		% these prevent an unwanted space from occurring between the last author name
		% and the end of the author line. i.e., if you had this:
		% 
		% \author{....lastname \thanks{...} \thanks{...} }
		%                     ^------------^------------^----Do not want these spaces!
		%
		% a space would be appended to the last name and could cause every name on that
		% line to be shifted left slightly. This is one of those "LaTeX things". For
		% instance, "\textbf{A} \textbf{B}" will typeset as "A B" not "AB". To get
		% "AB" then you have to do: "\textbf{A}\textbf{B}"
		% \thanks is no different in this regard, so shield the last } of each \thanks
		% that ends a line with a % and do not let a space in before the next \thanks.
		% Spaces after \IEEEmembership other than the last one are OK (and needed) as
		% you are supposed to have spaces between the names. For what it is worth,
		% this is a minor point as most people would not even notice if the said evil
		% space somehow managed to creep in.

	%	OTROS
		% The paper headers
% 			\markboth{Journal of \LaTeX\ Class Files,~Vol.~6, No.~1, January~2007}%
% 			{Shell \MakeLowercase{\textit{et al.}}: Bare Demo of IEEEtran.cls for Journals}
		% The only time the second header will appear is for the odd numbered pages
		% after the title page when using the twoside option.
		% 
		% *** Note that you probably will NOT want to include the author's ***
		% *** name in the headers of peer review papers.                   ***
		% You can use \ifCLASSOPTIONpeerreview for conditional compilation here if
		% you desire.
	
		% For peer review papers, you can put extra information on the cover
		% page as needed:
		 %\ifCLASSOPTIONpeerreview
		 %\begin{center} \bfseries EDICS Category: 3-BBND \end{center}
		 %\fi
		%
		% For peerreview papers, this IEEEtran command inserts a page break and
		% creates the second title. It will be ignored for other modes.
			%\IEEEpeerreviewmaketitle
	
		% If you want to put a publisher's ID mark on the page you can do it like
		% this:
		%\IEEEpubid{0000--0000/00\$00.00~\copyright~2007 IEEE}
		% Remember, if you use this you must call \IEEEpubidadjcol in the second
		% column for its text to clear the IEEEpubid mark.
	
		% use for special paper notices
		%\IEEEspecialpapernotice{(Invited Paper)}

	% make the title area
		\maketitle
		
		